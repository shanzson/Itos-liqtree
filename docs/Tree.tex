\documentclass{article}
\usepackage{amsmath}
\usepackage{amsthm}

\newtheorem{definition}{Definition}[section]

\newtheorem{theorem}{Theorem}[section]
\newtheorem{corollary}{Corollary}[theorem]
\newtheorem{lemma}[theorem]{Lemma}



\usepackage{graphicx}
\graphicspath{ {./images/} }

\setlength{\parskip}{4\smallskipamount}

\title{Ranged-Op Tree}
\author{Terence An}
\date{\today}

\begin{document}
\maketitle % creates title using information in preamble (title, author, date)

\begin{abstract}
  The paper gives theoretical and implementation details for a data structure
  that allows cheap queries of ranged sums (or other associative binary
  operation) and cheap insertion and deletion of
  new range values. Given a set $S$ of pairs $(l, r)$ where $l$ is a member of the
  monoid being summed and $r$ is itself a pair indicating a one-dimensional
  integer range, this tree is able to quickly query
  $min_{t \in R} [\sum_{s \in S : t \in s.r} s.l]$ for any given integer range
  $R$. The intention for designing such a data structure is to cheaply track AMM
  liquidity at every tick to provide the minimum liquidity available in a
  range. The implementation details will include portions specific to smart contract development.
\end{abstract}

\section{High Level}

Consider a sequence $a_i$$ that represents the amount of a resource available at a
given index $i$, we will interchangeably refer to this resource as
liquidity. Now we construct a binary tree such that each leaf node is indexed by
a unique $i$ and each node in the binary tree stores a numeric value. In order
to compute $a_i$ we have to sum the values in all the nodes on the path from the
$i$th leaf node to the root. This lets us query $a_i$ in logarithmic time, but
that is not the purpose of the tree. The purpose of the tree is to also answer range
based queries in logarithmic time. For example, users can query the total amount
of liquidity in a range, or the minimum, or the average, etc.

This tree can only compute functions that are additive. What I mean by this is that $f$ is
additive if it is composed of functions $g$ where $g$ follows the
property that
$\exists h$ s.t. $g(x_0 + k, x_1 + k, ..., x_n +k) = h(g(x_0, x_1, ..., x_n), k)$.
This property allows the value of $f$ to be computed by storing the values of
$g$ at each node for its subtree
and if any node modifies its value we can propogate the new values of $f$ up to
the root.

For example we can compute the minimum liquidity available in any range.
This is accomplished by storing two values at every node, the amount of
resources available for precisely the range represented by the node and the
subtree minimum. The subtree minimum answers the query if the query's
range exactly matches the node's range. If the ranges do not exactly match, the
queried range is broken up into partitions that do exactly match individual
nodes' ranges in the tree. Then we take the minimum over those nodes' subtree
minimums to answer the query.

In order to update the tree with resources for a given range, the range is once
again broken up into the smallest number of node ranges whose union matches the
given range, and that resource value is added to the sum in those nodes. Then
starting from each node, we walk up the tree propogating the new subtree
minimum.

Certain interesting properties make the walk cheap to accomplish. In fact
insertions, removals, and queries all take $O(log(n))$ where $n$ is the size of
the entire range covered by the tree.

\section{Theoretical Design}

The design starts similarly to an augmented tree where each node stores a value
$L$ and the minimum subtree value $M$, both of which are members of the monoid
being summed. Each node represents a range and has two children each
represeting half their range except for the leaf nodes which have no
children. Leaf nodes are those representing a single integer ranage.
The top node represents the full range. Note that not all ranges
are in the tree. We constrain ourselves to finite integer ranges and WLOG we
assume the finite rate is centered around 0 with a total range width that is a
power of two. Then the splits always divide the ranges in half which trivially makes all
ranges start with a power of two and have a width that is a power of two.

In order to query the tree with an range $R$ that is contained within our total
finite range, we decompose $R$ into $N_R = {R_0,R_1,...}$
where $N_R$ is a set of ranges possessed by exactly one node in the tree. These
ranges do not overlap, and $\bigcup N_R = R$. We will show the properties that
$N_R$ must follow. From now on, we assume $N_R$ is the smallest of such sets.

But first, we'll define left-adjacent and right-adjacent.
\begin{definition}[Left-adjacent Right-adjacent]
A node $p$ is
left-adjacent from $q$ if the path from either node's parent to the other node follows one left
branch first and then one or more right branches and the node closer to the root
is a right child. I.e. $lr+$ in regex notation
where $l$ is a taken left branch and $r$ is a taken right branch. Likewise, right-adjacent is
defined by a path with one right branch and then any number of left
branches where the lower depth node is a left child.
\end{definition}

\begin{definition}[Consecutive nodes]
We also define two nodes to be consecutive if the union of their ranges is
continguous, i.e. for all $a, b$ in the union of their ranges, any $c$ such that
$a < c < b$ must also be in the union.
\end{definition}

The main property will be building up to is this:

\begin{theorem}
  The nodes in $N_R$ follow a trapezoidal shape. Starting from the left-most
  node, there is a (possibly empty) consecutive series of left adjacent nodes of
  strict-monotonically lower depth
  (lower distance from the root), and starting from the right-most node there is
  another (possibly empty) consecutive series of right adjacent nodes of
  strict-monotonically lower depth. This left partition of nodes and right
  partition of nodes comprises the whole of $N_R$.
\end{theorem}

\begin{lemma}
  There is a unique $N_R$ that is minimal.
\end{lemma}
Assume there are two such $N_R$'s, $A$ and $B$. WLOG there exists a range $a$ in
$A$ that is not in $B$. We know ranges don't overlap and the unions are
identical thus there exists a set of ranges whose union overlaps with $a$
identically. Replacing them with $a$ in $B$ produces a smaller $N_R$.

\begin{lemma}
  If one node's range is a subset of another node, then that node is a
  descendent of the other node.
\end{lemma}
This follows trivially from the construction of the tree.

\begin{lemma}
  There cannot be two nodes of the same depth that overlap in range.
\end{lemma}
By construction nodes of the same depth have the same range width and their
starting indices are all multiples of that width. If two node ranges overlapped
the start index of one of them cannot be a multiple of the width.

\begin{lemma}
  $N_R$ nodes do not overlap in range.
\end{lemma}
Node ranges start with a power of two. If a node starts within the range of
another node, then that node's range is entirely contain within the other node's
range or there is a subsequent section that isn't. In the
former case, it means the node is not necessary and $N_R$ is not minimal. In the
latter case these two nodes cannot be of the same depth by the previous
lemma. But the deeper of the two nodes must have a parent at the same depth at
the other node which is not the other node (or else it'd be fully contained) and
that parent's range must also overlap which cannot be true by the previous lemma.

\begin{corollary}
  There is a well ordering of $N_R$ nodes by ranges where every end index of one
  range precedes the next smallest starting index whose corresponding end index is
  the next smallest end index.
\end{corollary}

\begin{lemma}
  When $N_R$ are well-ordered, all nodes are consecutive with their neighbors in the ordering.
\end{lemma}
If they were not, then $\bigcup N_R \neq R$ since $R$ is continuous.

\begin{definition}[Far-adjacent]
  Let $p$ be lowest common ancestor of two nodes. These two nodes are
  far-adjacent if the right child of $p$ is
  left-adjacent to one node and the left child of
  $p$ is right-adjacent to the other.
\end{definition}

\begin{theorem}
  Two nodes are consecutive if and only if they are either siblings,
  left-adjacent, right-adjacent, or far-adjacent.
\end{theorem}

\begin{lemma}
  If $a, b, c$ are consecutive nodes and $a$ and $b$ are left-adjacent, then $b$
  and $c$ are either left-adjacent or far-adjacent.
\end{lemma}

\begin{lemma}
  If $a, b, c$ are consecutive nodes and $b$ and $c$ are right-adjacent, then
  $a$ and $b$ are either right-adjacent or far-adjacent.
\end{lemma}

\begin{lemma}
  The left-most node is a right child and the right-most node is a left child.
\end{lemma}

\begin{theorem}
  $N_R$ when well ordered, is a sequence of left-adjacent nodes, a pair of
  far-adjacent nodes, and then a sequence of right-adjacent nodes.
\end{theorem}

\begin{lemma}
  Left-adjacent nodes cannot be of the same depth. Likewise for right-adjacent nodes.
\end{lemma}

Show they decrease in depth.

We will call the lowest depth node of a set (the node closest to root) the
tallest node.

The tallest nodes of the two sides are far-adjacent.

Theorem proved.

This provides us with the first property of the tree, the range partition.

Then we use the range partition to add liquidity to and walk up the tree.

Prove this update to minimim will also give the subtree minimum value for any
contained index.

This naturally holds for removing liquidity.

Also holds for tracking maximums.

\begin{definition}
  The peak of our breakdown is the highest node in the smallest subtree
  containing our breakdown trapezoid.
\end{definition}

\begin{theorem}
  The longest common prefix of the bit representations of the two range bounds is the
  bit representation of the peak's base, and the least significant bit of the
  base is the peak's range.
\end{theorem}

\begin{proof}
  First, this procedure produces a parent of both the low and high bounds. The
  produced key's range spans both the low and high since it spans all possible
  values in that range, which includes those two since a bit sequence is
  equivalent to a series of traversals in the tree. Since
  all nodes at a single tier have mutually exclusive ranges this is the only
  possible parent at that tier.

  Second, the bit below the lsb is a 0 in the low's bit representation and a 1
  in the high's representation or else the xor would have produced that below or
  lower as its lsb. Thus no range smaller than the one we've found is sufficient
  to cover the whole range. Therefore lowest common ancestor must be at this
  tier.

  Thus this procedure produces the lowest common ancestor.
\end{proof}


\section{Implementation}

\subsection{Node Key Semantics}

A node is keyed by a int48 which is the external product of the range and the
base in that order. The top 24 bits is the range and the bottom is the base.
The base is literally the start of the range that node ``covers'' and the range
is the size of its ``covering''. In other words, any value stored at the node
with base $B$ and range $R$ applies that value to all ticks in $[B, B+R)$.

\subsection{Range Semantics}

A range is specified by a low and a high. The range of integer values it
represents is $[low, high)$. From our theorems above, we know that low will
always represent a right node. High will also represent a right node in this
case because it is exclusive. This makes the high point to the sibling of the
right base of our range trapezoid, which is a left node.

The reason high is exclusive is because right nodes' least significant
bit is the bit representation of that node's range. By xoring that range with
high we recover the base value for the left base node of our range trapezoid.
Combining this base with the range we have gives the proper Node Key for the
left base.

However, so far we have only described scenarios where we have a right base and
a left base for our trapezoid. That won't always be the case depending on the
range chosen. The leftmost node of our trapezoid could be a left node in which
case we don't have a left base. And likewise for right nodes and the right base.
When we have a single base node, the proper range to supply would then be the
lowest ancestor of all nodes in our range breakdown. This works conveniently
with our peak calculation. Recall that the peak is calculated by finding the
lowest common ancestor of our base nodes. Calculating peak between a node and
its ancestor will always return the ancestor thus letting us avoid calculating
one side of the trapezoid by checking if the respective range bound is equal to
peak.

\subsection{Value Propogation}

When we update the values held in any node, we need to propogate it up. How we
propogate is documented in the theorems, but here we'll discuss how we can do so
cheaply in our implementation. To walk up a left branch, we simply shift the
range one to the left and the parent node has the same base. To walk up a right
branch we xor the range with our base to get the parent's base and shift the
range one to the left. An xor of the range with our key will always give the
correct key for our sibling. In this way, given a node, we can easily find its parent and
update their values with the values of the given node and its sibling.

\subsection{Range Breakdown Traversal}

When we query over a range, we need to walk over all nodes of our range
breakdown. We can follow the procedure from value propogation to visit all these
nodes, but there is a cheaper method. In value propogation we must visit
intermediary nodes between our breakdown to update their values, but here we
have no need.

When starting at a right node, we know the next node in our breakdown is left
adjacent meaning we want to find the sibling of the first left node in the path
of our ancestors starting from ourselves. The series of edges in finding the
first ancestral left node is all right edges before one final left edge. This
means that the bits in our representation starting at the bit in our range bit's
index will be a series of 1 until 0 when read to the left. Thus we simply add 1
to turn all of those into a 0 and the final 0 into a 1. This gives us the exact
node we were looking for. Just clearing all the zeroes gets us to the first
ancestral left node, and the addition of a 1 gives us that node's sibling at the
correct range.

When starting at a left node the situation is the opposite. We have a series of
0 bits and then a 1 bit which we'll want to convert to a 0 bit to give us the
next left node in our breakdown. To do this we simply start with finding the
least significant bit of our base to get the next range we'll want, and then by
xoring that range with our base we can get the base of the next breakdown node.

To know when to stop traversing, we first calculate the peak and break when our
next node crosses the peak. Conceptually this makes sense, but in practice
recall that the peak's base is actually the start of the range which covers our
own trapezoid so we can't determine if we've cross the peak just by comparing
base values. Instead, we notice that the left adjacent node of our final left
side node will be the right child of our peak and the right adjacent node of our
final right side node will be the left child of our peak. Thus stopping iteration
when we visit a node who's range is one below our peak's range achieves what we
want.

\end{document}
